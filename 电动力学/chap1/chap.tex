\ifx\allfiles\undefined
\documentlecture[12pt, a4paper, oneside, UTF8]{ctexbook}  %  这一句是新增加的
\usepackage[dvipsnames]{xcolor}
\usepackage{amsmath}   % 数学公式
\usepackage{graphicx}
\usetikzlibrary{arrows, calc, decorations.pathmorphing}
\allowdisplaybreaks % 允许公式跨页换行
\newcommand{\pa}{\partial}
\newcommand{\mathminus}{\!\!-\!\!} % 数学环境连字符
\newcommand{\vsup}[1]{\raisebox{-0.1ex}{$\scriptstyle #1$}}
\newcommand{\lsup}[1]{\raisebox{-0.85ex}{$\scriptstyle #1$}}

\begin{document}
%\input{/config/cover} % 单独编译时,其实不用编译封面目录之类的,如需要不注释这句即可
\else
\fi
%  ↓↓↓↓↓↓↓↓↓↓↓↓↓↓↓↓↓↓↓↓↓↓↓↓↓↓↓↓ 正文部分
\section{真空中的麦克斯韦方程组}
\begin{add}
    \begin{enumerate}
        电磁学的四个实验定律
        \item \textbf{库仑定律 (Coulomb's Law)}:
        \begin{equation}
        \vecbold{F} = \frac{q_1 q_2}{4 \pi \epsilon_0} \frac{\vecbold{r}_1 - \vecbold{r}_2}{|\vecbold{r}_1 - \vecbold{r}_2|^3}
    \end{equation}
    
        \item \textbf{毕奥-萨伐尔定律 (Biot-Savart Law)}:
        \begin{equation}
        d\vecbold{B}(\vecbold{r}_1) = \frac{\mu_0 I_2}{4\pi} d\vecbold{l}_2 \times \frac{\vecbold{r}_1 - \vecbold{r}_2}{|\vecbold{r}_1 - \vecbold{r}_2|^3}
    \end{equation}    
        \item \textbf{法拉第电磁感应定律 (Faraday's Law of Induction)}:
        \begin{equation}
        \mathcal{E} = -\frac{d\Phi_B}{dt}, \quad \text{其中} \quad 
        \mathcal{E} = \oint_C \vecbold{E} \cdot d\vecbold{l}, \quad 
        \Phi_B(t) = \int_S \vecbold{B}(\vecbold{r}, t) \cdot d\vecbold{S}
    \end{equation}
    
        \item \textbf{安培力定律 (Ampère's Force Law)}:
        \begin{equation}
        d\vecbold{F} = I \, d\vecbold{l} \times \vecbold{B}
    \end{equation}
    \end{enumerate}
\end{add}
\subsection{静电现象}
\begin{defn}
    电场强度

    两个静止不动的电荷 \( q_1 \) 和 \( q_2 \) 之间的受力由库仑定律
    \[
    \vecbold{F} = \frac{q_1 q_2}{4 \pi \epsilon_0} \frac{\vecbold{r}_1 - \vecbold{r}_2}{|\vecbold{r}_1 - \vecbold{r}_2|^3}
    \]
给出。若我们定义
\begin{gather*}
        \vecbold{E}(\vecbold{r}_1) = \frac{q_2}{4 \pi \epsilon_0} \frac{\vecbold{r}_1 - \vecbold{r}_2}{|\vecbold{r}_1 - \vecbold{r}_2|^3}
        =\frac{q_2}{4 \pi \epsilon_0} \frac{\vecbold{r}_1 - \vecbold{r}_2}{|\vecbold{r}_1 - \vecbold{r}_2|^2}\vecbold{e}_{(\vecbold{r}_1,\vecbold{r}_2)}\\
        \vecbold{F} = q_1 \vecbold{E}(\vecbold{r}_1)
\end{gather*}
实验结果显示电场具有叠加性。
任取 \(\Omega\) 内的一点 \(\vecbold{r} = (x,y,z)\),定义 \(dQ = \rho(\vecbold{x}, \vecbold{y}, \vecbold{z}) dV\) 
为以 \(\vecbold{r}\) 为中心的一个小的球形体积元 \(dV\) 内的电荷量。因此
\[
\vecbold{E}(\vecbold{r}) = \frac{1}{4\pi\epsilon_0}\int_{\Omega}\rho(x,y,z)\frac{\vecbold{r} - \vecbold{r}}{|\vecbold{r} - \vecbold{r}|^3}\,dxdydz
\]
\end{defn}
\begin{defn}
    电场对于任一闭合曲面\(S\)的通量
\begin{align*}
    \oint_S \vecbold{E} \cdot d\vecbold{S} 
    &= \oint_S \frac{q}{4\pi \epsilon_0} \frac{\vecbold{r} - \vecbold{r}_0}{|\vecbold{r} 
    - \vecbold{r}_0|^3} \cdot d\vecbold{S} 
    = \oint_S \frac{q}{4\pi \epsilon_0} d\Omega\\
    &= \frac{q}{4\pi \epsilon_0} \oint_S d\Omega 
    = \frac{q}{4\pi \epsilon_0} 4\pi = \frac{q}{\epsilon_0}\\
\oint_S \vecbold{E} \cdot d\vecbold{S} 
&= \oint_S (\vecbold{E}_1 + \vecbold{E}_2 + \cdots + \vecbold{E}_N) \cdot d\vecbold{S}\\
&= \oint_S \vecbold{E}_1 \cdot d\vecbold{S} 
+ \oint_S \vecbold{E}_2 \cdot d\vecbold{S} 
+ \cdots + \oint_S \vecbold{E}_N \cdot d\vecbold{S}\\
&= \frac{q_1}{\epsilon_0} + \frac{q_2}{\epsilon_0} 
+ \cdots + \frac{q_N}{\epsilon_0} 
= \frac{1}{\epsilon_0} \sum_{i=1}^N q_i = \frac{Q}{\epsilon_0}\\
\oint_S \vecbold{E} \cdot d\vecbold{S} &= \frac{Q}{\epsilon_0} 
= \frac{1}{\epsilon_0} \int_{\Omega} \rho(x, y, z) dx dy dz
\end{align*}
\end{defn}
\begin{thm}
    电场的散度---高斯定理
    \begin{tui}
        以空间中一点 \(\vecbold{r} = (x, y, z)\) 为中心,半径为 \(\delta\) 的一个小的球形区域 \(\Omega_\delta\) 上。我们有
        \[\oint_{S_{\delta}} \vecbold{E} \cdot d\vecbold{S} = \frac{1}{\epsilon_0} \int_{\Omega_{\delta}} \rho(x, y, z) dx dy dz = \int_{\Omega_{\delta}} \operatorname{div}\vecbold{E}(x, y, z) dx dy dz
        \]
        令\(\delta\to 0\) 
        \begin{align*}
        \frac{1}{\epsilon_0} \int_{\Omega_\delta} \rho(x, y, z) dx dy dz 
        &= \frac{1}{\epsilon_0} \int_{\Omega_\delta} \rho_0(x, y, z) dx dy dz
        = \frac{1}{\epsilon_0} \rho_0(x, y, z) \frac{4\pi}{3} \delta^3 \\
        =\int_{\Omega_\delta} 
        \operatorname{div} \vecbold{E}(x, y, z) dx dy dz 
        &= \int_{\Omega_\delta} 
        \operatorname{div} \vecbold{E}_0(x, y, z) dx dy dz 
        = \operatorname{div} \vecbold{E}_0(x, y, z) \frac{4\pi}{3} \delta^3
        \end{align*}
        \begin{equation}\label{f1}
        \operatorname{div} \vecbold{E}(x,y,z) \equiv \nabla \cdot \vecbold{E}(\vecbold{r}) = \frac{1}{\epsilon_0} \rho(x,y,z)
        \end{equation}
        这是电磁学中高斯定理的微分形式,也是真空中的麦克斯韦方程组的第一个方程,
        由高斯(Karl Friedrich Gauss)于1839年在研究地磁场分布的过程中所证明。
    \end{tui}
\end{thm}
\begin{thm}
    静电场的旋度---安培环路定理
    \begin{align*}
        \vecbold{E} ( \vecbold{r} ) &= \frac{1}{4 \pi \varepsilon_0} 
        \int \frac{\rho ( \vecbold{r} ) d \tau}{R^3} \vecbold{R}
        = -\frac{1}{4\pi\varepsilon_0} \int \rho(\vecbold{r}) d\tau 
        \cdot \nabla\left( \frac{1}{R} \right) \\
        &= - \nabla \left[ \frac{1}{4\pi\varepsilon_0} 
        \int \frac{\rho(\vecbold{r})}{R} d\tau \right] 
        = - \nabla \varphi(\vecbold{r})\quad \varphi(\vecbold{r})\text{为标量势}\\
        \nabla \times \vecbold{E}(\vecbold{r}) 
        &= -\nabla \times \nabla \varphi(\vecbold{r}) \equiv 0
    \end{align*}
    这是真空中的麦克斯韦方程组的第二个方程。
\end{thm}
\begin{lemma}
    \begin{align*}
        \nabla r&=\left(\frac{\partial}{\partial x} \sqrt{x^2 + y^2 + z^2},
        \frac{\partial}{\partial y} \sqrt{x^2 + y^2 + z^2},
        \frac{\partial}{\partial z} \sqrt{x^2 + y^2 + z^2}\right)\\
        &= \left(\frac{x}{\sqrt{x^2 + y^2 + z^2}},
        \frac{y}{\sqrt{x^2 + y^2 + z^2}},
        \frac{z}{\sqrt{x^2 + y^2 + z^2}}\right)
        =\left(\frac{x}{r},\frac{y}{r},\frac{z}{r}\right)=\frac{\vecbold{r}}{r}\\
        \nabla \left( \frac{1}{R} \right) &= - \frac{\nabla R}{R^2} = - \frac{1}{R^2} \frac{\vecbold{R}}{R} 
        = - \frac{\vecbold{R}}{R^3}
    \end{align*}
\end{lemma}
\begin{defn}
    电偶极子
    \begin{align*}
\varphi(\vecbold{r}) &= \frac{q}{4\pi\varepsilon_0} 
\left( \frac{1}{r_+} - \frac{1}{r_-} \right)
\approx \frac{q}{4\pi\varepsilon_0} \frac{r_- - r_+}{r^2}
\approx \frac{q}{4\pi\varepsilon_0} \frac{\cos\theta}{r^2}\\
&= \frac{p \cos\theta}{4\pi\varepsilon_0 r^2}
= \frac{\vecbold{p} \cdot \vecbold{r}}{4\pi\varepsilon_0 r^3}\\
\vecbold{E}(\vecbold{r}) &= -\nabla \varphi
= -\frac{1}{4\pi\varepsilon_0} \left[ \frac{\nabla (\vecbold{p} \cdot \vecbold{r})}{r^3} + \vecbold{p} 
\cdot \vecbold{r} \nabla \frac{1}{r^3} \right]
= -\frac{1}{4\pi\varepsilon_0} \left[ \frac{\vecbold{p} - 3(\vecbold{p} 
\cdot \hat{\vecbold{r}})\hat{\vecbold{r}}}{r^3} \right]
    \end{align*}
\end{defn}
\subsection{静磁现象}
\begin{defn}
    电流、电流密度、电荷守恒
\begin{itemize}
    \item \textbf{电流}:单位时间内垂直穿过某一特定截面的电荷量,用 \( I \) 表示。
    \[I = \frac{\Delta q}{\Delta t} \bigg|_S\]
    \item \textbf{电流密度}: \( \vecbold{j} \) 为单位面积单位时间通过的电荷量:
    \(\displaystyle \vecbold{j} = \frac{\Delta q}{\Delta t \Delta S}\)
\[
j = \frac{\rho \Delta \Omega}{\Delta t \Delta S} 
= \frac{\rho \Delta S v \Delta t}{\Delta t \Delta S} = \rho v
\Rightarrow \vecbold{j} = \rho(\vecbold{r}) \vecbold{v}(\vecbold{r}) 
\Rightarrow I = \int_S \vecbold{j} \cdot d\vecbold{S}
\]
\item \textbf{电荷守恒}:实验表明电荷是守恒的。
\[
\oint_{S} \vecbold{j} \cdot d\vecbold{S} 
= -\frac{d}{dt} \int_{V} \rho \, d\tau
\Rightarrow \int_V\left(\nabla\cdot \vecbold{j}
+\frac{\pa \rho}{\pa t}\right)\,d\tau =0
\Rightarrow \nabla\cdot \vecbold{j}+\frac{\pa \rho}{\pa t}=0
\]
\end{itemize}
\end{defn}
\begin{law}
    安培定律(续)
    
    若真空中的两个电流元 \(\vecbold{j}_1 d \tau_1\) 和 \(\vecbold{j}_2 d \tau_2\),
    则安培定律告诉我们 \(2\) 对 \(1\) 的作用力 \(d \vecbold{F}_{12}\) 为
\[
d \vecbold{F}_{12} = \frac{\mu_0}{4 \pi} \frac{\vecbold{j}_1 d \tau_1 \times (\vecbold{j}_2 d \tau_2 \times \vecbold{R}_{12})}{\vecbold{R}_{12}^3}
\]
\end{law}
\begin{example}
    电流元之间的相互作用力不满足牛顿的作用力与反作用力定律,即 \( d\vecbold{F}_{12} \neq d\vecbold{F}_{21} \)(比如考虑下图的情况)。
\[
\uparrow \vecbold{j}_1 d\tau_1
\qquad\qquad
\underset{\vecbold{j}_2 d\tau_2}{\longrightarrow}
\]
简单的回应是因为不可能存在稳定的电流元,实验所能测量的只能是闭合回路的情况。
\begin{explain}
    考虑两闭合载流线圈,则2对1的作用力为
\begin{align*}
    \vecbold{F}_{12} &= \frac{\mu_0 I_1 I_2}{4\pi} 
    \oint_{l_1} \oint_{l_2} \frac{d\vecbold{l}_1 
    \times (d\vecbold{l}_2 \times \vecbold{R}_{12})}{\vecbold{R}_{12}^3}\\
    &= \frac{\mu_0 I_1 I_2}{4\pi} 
    \oint_{l_1} \oint_{l_2} \left[ \frac{d\vecbold{l}_2 (d\vecbold{l}_1 
    \cdot \vecbold{R}_{12})}{\vecbold{R}_{12}^3} 
    - \frac{\vecbold{R}_{12} (d\vecbold{l}_2 \cdot d\vecbold{l}_1)}
    {\vecbold{R}_{12}^3} \right]\\
    &= -\frac{\mu_0 I_1 I_2}{4\pi} \oint_{l_2} d\vecbold{l}_2 
    \oint_{l_1} \left[ d\vecbold{l}_1 \cdot \nabla \frac{1}{R_{12}} \right] 
    - \frac{\mu_0 I_1 I_2}{4\pi} \oint_{l_1} \oint_{l_2} 
    \frac{\vecbold{R}_{12} (d\vecbold{l}_2 \cdot d\vecbold{l}_1)}
    {\vecbold{R}_{12}^3} \\
    &= 0 + -\frac{\mu_0 I_1 I_2}{4\pi} \oint_{l_1} 
    \oint_{l_2} \frac{\vecbold{R}_{12} (d\vecbold{l}_2 \cdot d\vecbold{l}_1)}
    {\vecbold{R}_{12}^3} 
    = -\vecbold{F}_{21}
\end{align*}
\end{explain}
\end{example}
\begin{law}
    毕奥-萨伐尔定律(续)
    
    将作用在电流元 \( \vecbold{j}_1 d\tau_1 \) 上的力写为
\[
d\vecbold{F}_1 = \vecbold{j}_1 d\tau_1 \times \vecbold{B}(\vecbold{r})
\]
其中 \(\displaystyle \vecbold{B}(\vecbold{r}) = \frac{\mu_0}{4\pi} \vecbold{j}_2 d\tau_2 
\times \frac{\vecbold{R}_{12}}{R_{12}^3}\) 为电流元 \( \vecbold{j}_2 d\tau_2 \) 
在 \(\vecbold{r}\) 处产生的磁场。由叠加原理,对任意的电流分布 \(\vecbold{j}(\vecbold{r})\),
其在 \(\vecbold{r}\) 处产生的磁场为
\[
\vecbold{B}(\vecbold{r}) = \frac{\mu_0}{4\pi} 
\int_{V'} \frac{\vecbold{j}(\vecbold{r}) d\tau \times \vecbold{R}}{R^3}
=\frac{\mu_0}{4\pi} 
\oint_{L'} I' \, d\vecbold{l}'\times \frac{\vecbold{R}}{R^3}
\]
函数 \(\vecbold{B}(\vecbold{r})\) 称为磁感应强度
\end{law}
\begin{thm}
    磁场的散度
    \begin{align*}
        \vecbold{B}(\vecbold{r}) &= \frac{\mu_0}{4\pi} \int \vecbold{j}(\vecbold{r}') \times \frac{\vecbold{R}}{R^3} d\tau' 
        = -\frac{\mu_0}{4\pi} \int \vecbold{j}(\vecbold{r}') \times \left( \nabla \frac{1}{R} \right) d\tau' 
        = \frac{\mu_0}{4\pi} \int \left( \nabla \frac{1}{R} \right) \times \vecbold{j}(\vecbold{r}') d\tau' \\
        &= \frac{\mu_0}{4\pi} \int \nabla \times \frac{\vecbold{j}(\vecbold{r}')}{R} d\tau' 
        = \nabla \times \left[ \frac{\mu_0}{4\pi} \int \frac{\vecbold{j}(\vecbold{r}')}{R} d\tau' \right] 
        = \nabla \times \vecbold{A}
    \end{align*}
    其中,矢势 \(\vecbold{A}\) 定义为:
\(\displaystyle
\vecbold{A}(\vecbold{r}) = \frac{\mu_0}{4\pi} \int \frac{\vecbold{j}(\vecbold{r}')}{R} d\tau'
\)
\[
\nabla \cdot \vecbold{B}(\vecbold{r}) 
= \nabla \cdot (\nabla \times \vecbold{A}) = 0
\Rightarrow \nabla \cdot \vecbold{B}(\vecbold{r},t)=0 \Rightarrow
\oint_S d\vecbold{S}\cdot\vecbold{B}(\vecbold{r},t)=0
\]
\end{thm}
\begin{thm}
    磁场的旋度
    \[
\nabla \times \vecbold{B}(\vecbold{r}) = \nabla \times (\nabla \times \vecbold{A}) = \nabla(\nabla \cdot \vecbold{A}) - \nabla^2 \vecbold{A}
\]
\begin{align*}
\nabla \cdot \vecbold{A} &= \nabla \cdot \frac{\mu_0}{4\pi} 
\int_V \frac{\vecbold{j}(\vecbold{r}')}{R} d\tau' 
= \frac{\mu_0}{4\pi} \int_{V'} \left( \nabla \frac{1}{R} \right) 
\cdot \vecbold{j}(\vecbold{r}') d\tau'\\
    &=-\frac{\mu_0}{4\pi} \int_{V'} \left( \nabla' \frac{1}{R} \right) 
    \cdot \vecbold{j}(\vecbold{r}') d\tau'\\
    &=-\frac{\mu_0}{4\pi} \int_{V'} \nabla'\cdot\left(  \frac{\vecbold{j}(\vecbold{r}') }{R} \right) d\tau'
    +\frac{\mu_0}{4\pi} \int_{V'} \frac{1}{R}
    \nabla'\cdot \vecbold{j}(\vecbold{r}') d\tau'\\
    &\because\quad \nabla'\cdot \vecbold{j}(\vecbold{r}') =-\frac{\pa \rho}{\pa t}=0\\
    \nabla \cdot \vecbold{A} &=
    -\frac{\mu_0}{4\pi} \int_{V'} \nabla'\cdot\left(  \frac{\vecbold{j}(\vecbold{r}') }{R} \right) d\tau'
    =-\frac{\mu_0}{4\pi} \oint_{S'} \,d S'\cdot\frac{\vecbold{j}(\vecbold{r}') }{R}
\end{align*}
\begin{zhu}
\[
\left.
\nabla f(R) = \frac{\partial f}{\partial R} \nabla R = \frac{\partial f}{\partial R} \frac{(\vecbold{r} - \vecbold{r}')}{R}\\
\nabla' f(R) = \frac{\partial f}{\partial R} \nabla' R = \frac{\partial f}{\partial R} \frac{(\vecbold{r}' - \vecbold{r})}{R}
\right\}
\Rightarrow \nabla f(R) = -\nabla' f(R)\]

\end{zhu}
\end{thm}
\begin{example}
    以速度 \( \vecbold{v} \) 运动的电荷 \( q \) 产生的电流密度为  
\[
\vecbold{j} = q\vecbold{v}\delta(\vecbold{r} - vt\hat{\vecbold{x}}) \quad (\text{仅在 } v \ll \text{光速时成立})
\]

因此其在 \(\vecbold{B}\) 场中所受的力为  
\[
\vecbold{F} = \int_{\tau} q\delta(\vecbold{r} - vt\hat{\vecbold{x}})d\tau \vecbold{v} 
\times \vecbold{B} = q\vecbold{v} \times \vecbold{B}
\]

若空间既有磁场又有电场,则总受力为  
\[
\vecbold{F} = q\left( \vecbold{E} + \vecbold{v} 
\times \vecbold{B} \right)
\]

这就是描述带电粒子在空间既有电场又有磁场时的受力---Lorentz 力。
\end{example}




\begin{law}
    电动力学中的欧姆定律(微分形式)和电磁学中的欧姆定律(积分形式)本质上是相同的。

    微分形式
\(
\vecbold{J} = \sigma \vecbold{E}
\)
,积分形式
\(
V = IR
\)
    \begin{proof}
    对于一段均匀导体,假设其长度为 \(L\),横截面积为 \(A\),电导率为 \(\sigma\),则电阻 \(R\) 可以表示为:
\[
R = \frac{L}{\sigma A}
\]

电场强度 \(\vecbold{E}\) 与电压 \(V\) 的关系为:
\[
V = \int \vecbold{E} \cdot d\vecbold{l} = EL
\]

电流密度 \(\vecbold{J}\) 与电流 \(I\) 的关系为:
\[
I = \int \vecbold{J} \cdot d\vecbold{A} = JA
\]

将微分形式的欧姆定律 \(\vecbold{J} = \sigma \vecbold{E}\) 代入,可以得到:
\[
I = \sigma E A = \sigma \left(\frac{V}{L}\right) A
\]

整理后得到:
\[
V = I \left(\frac{L}{\sigma A}\right) = IR
\]
这正是积分形式的欧姆定律。
    \end{proof}
\end{law}






%    请你转换为latex源码,矢量用\vecbold表示,使用align*或gather*






%  ↑↑↑↑↑↑↑↑↑↑↑↑↑↑↑↑↑↑↑↑↑↑↑↑↑↑↑↑ 正文部分
\ifx\allfiles\undefined
\end{document}
\fi
